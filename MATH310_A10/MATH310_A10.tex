\documentclass[11pt]{article}
\usepackage{algorithm}
\usepackage{algpseudocode}
\usepackage{amssymb,latexsym,amsmath,amsthm,graphicx, cite}
 \author{Chris Camano: ccamano@sfsu.edu}
 \title{MATH310  Homework 10 }
 \date

\usepackage{mathptmx}
\usepackage{pgfplots}
\usetikzlibrary{positioning}
\usepackage{multirow}
\usepackage{float}
\restylefloat{table}
\hoffset=0in
\voffset=-.3in
\oddsidemargin=0in
\evensidemargin=0in
\topmargin=0in
\textwidth=6.5in
\textheight=8.8in
\marginparwidth 0pt
\marginparsep 10pt
\headsep 10pt

\theoremstyle{definition}  % Heading is bold, text is roman
\newtheorem{theorem}{Theorem}
\newtheorem{corollary}{Corollary}
\newtheorem{definition}{Definition}
\newtheorem{example}{Example}
\newtheorem{proposition}{Proposition}



\newcommand{\Z}{\mathbb{Z}}
\newcommand{\N}{\mathbb{N}}
\newcommand{\Q}{\mathbb{Q}}
\newcommand{\R}{\mathbb{R}}
\newcommand{\C}{\mathbb{C}}

\newcommand{\lcm}{\mathrm{lcm}}


\usepackage[many]{tcolorbox}
\usepackage{lipsum}
\tcbset{breakable}

\newcommand{\bigline}{\\\noindent\makebox[\linewidth]{\rule{\paperwidth}{0.4pt}}\\}
\newcommand{\block}[2]{\begin{tcolorbox}[title={#1}]{#2}\end{tcolorbox}}
\setlength{\parskip}{0cm}
%\renewcommand{\thesection}{\Alph{section}}
\renewcommand{\thesubsection}{\arabic{subsection}}
\renewcommand{\thesubsubsection}{\arabic{subsection}.\arabic{subsubsection}}
\bibliographystyle{amsplain}

%\input{../header}

\begin{document}
\maketitle
\block{Stein 1.13}{
\begin{enumerate}
  \item Prove that if a postive integer n is a perfect square, then n cannot be written in the form 4k+4 for k an integer.
  \item Prove that no integers in the sequence:
  \[
    11,111,1111,\cdots
  \]
  is a perfect square
\end{enumerate}
}
\begin{proof}
if n is of the form: $4k+3$ we can show that n must then be odd since :
\[
  4k+3=2(2k+1)+1
\]
so if there exists a value x such that $x^2=4k+3$ then we can show by the definition of an odd number for an odd number of the form : $(2l+1)$that :
\begin{align*}
  &(2l+1)^2=4k+3\\
  &4l^2+4l+1=4k+3\\
  &4l^2+4l-4k+1=3\\
  &4(l^2+l-k)+1=3\\
  &4(l^2+l-k)=2\\
  &2(l^2+l-k)=1\\
  &(l^2+l-k)=\frac{1}{2}
\end{align*}
which is a contradiction since k and l are integers .
\end{proof}
\begin{proof }
  Since numbers of the form 111... have remainder 3 mod 4 it can be shown that by the first proof since numbers of the form 4k+3 are never perfect squares that the sequence contains no perfect squares. \\
  Below is the proof :
let the elements of the sequence be generated by the following expression:
\[
  a_n=\sum_{i=0}^n10^i
\]
Since 4 divides all values of this sequence for $i>2$ we can consider the following:
\[
  \sum_{i=0}^n10^i=11+\sum_{i=2}^n\equiv 11\mod 4
\]
Which is the same as :
\[
  11\equiv 3 \mod 4
\]
and by the proof of part a we know that numbers of the forma 4k+3 cannot be perfect squares Thus we conclude that no elements of the sequence are perfect sqaures.
\end{proof }
%%%%%%%%%%%%%%%%%%%%%%%%%%%%%%%%%%%%%%%%%%%%%%%%%%
\block{Andrews 5.1.3}{
Find $\tilde{a}$, the inverse of a modulo c, when :
\begin{enumerate}
  \item $a=2$ and $c=5$\\
  \item $a=7$ and $c=9$\\
  \item $a=12$ and $c=17$\\
\end{enumerate}
}
\begin{proof}
Since the values of a and c are relatively small most of these were computed by inspection, we Know that there exists an inverse if a and c are coprime and in this case all a and c are in fact coprime meaning its worthwhile to consider potential solutions.
\begin{enumerate}
  \item $a=2$ and $c=5$\\
  The modular inverse of a is 3 since:
  \[
    6\equiv 1\mod 5
  \]
  \item $a=7$ and $c=9$\\
  The modular inverse of a is 4 since:
  \[
    28\equiv 1\mod 9
  \]
  \item $a=12$ and $c=17$\\

  The modular inverse of a is 10 since:
  \[
    120\equiv 1\mod17
  \]
\end{enumerate}
\end{proof}
%%%%%%%%%%%%%%%%%%%%%%%%%%%%%%%%%%%%%%%%%%%%%%%%%%
\block{Andrews 5.2.5}{
What is the remainder when $41^{75}$ is divided by 3?
}
\begin{proof}
Using Fermat's little theorem we know that:
\[
  n^{p-1}\equiv 1\mod p
\]
in our context:
\[
  n^2\equiv 1 \mod 3
\]
Leveraging this fact we can re-express our original problem as follows:
\begin{align*}
  &41^{75}\equiv x \mod 3\\
  &41(41^{37})^2\equiv x \mod 3\\
  &41\equiv x \mod 3 \\
  &41^{75}\equiv41\equiv 2 \mod 3 \\
\end{align*}

\end{proof}
%%%%%%%%%%%%%%%%%%%%%%%%%%%%%%%%%%%%%%%%%%%%%%%%%%
\block{Andrews 5.2.6}{
What is the remainder when $473^{38}$ is divided by 5?
}
\begin{proof}
  Again Using Fermat's little theorem we know that:
  \[
    n^{p-1}\equiv 1\mod p
  \]
  in our context:
  \[
    n^4\equiv 1 \mod 5
  \]
  Leveraging this fact we can re-express our original problem as follows:
  \begin{align*}
    &473^{38}\equiv x \mod 5\\
    &473^{2}(473^{4})^9\equiv x \mod 5\\
    &473^{2}\equiv x \mod 5 \\
  \end{align*}
  $473^2$ is a fairly large number but luckily since we are working mod 5 we only need to consider the final digit since the rest will be divisible by 5. giving:
  \[
    473^{38}\equiv 473^{2}\equiv 3^2 \equiv9\equiv 4 \mod 5
  \]
\end{proof}
%%%%%%%%%%%%%%%%%%%%%%%%%%%%%%%%%%%%%%%%%%%%%%%%%%
\block{Problem 5}{
If n is composite then $2^n-1$ is composite. Compute using computational methods 3 numbers of the form $2^n-1$ that are prime
}
\begin{proof}
  Let n be composite, then n-= $kp$ where p is a prime
  \begin{align*}
    &2^n-1=2^{kp}-1\\\\
    &(2^k)^p-1\\\\
    &\frac{(2^k)^p-1}{2^k-1}(2^k-1)\\\\
  \end{align*}
  Note that this is the result of a finite geometric series:
  \[
    \left[\sum_{i=1}^p(2^k)^{p-i}\right](2^k-1)
  \]
  Both of these are integers so we have shown that $2^n-1$ is a composite number since it is the produc tof two integers


\end{proof}
\begin{proof}
  Numbers of the form $2^n-1$ that are prime are denoted Mersenne primes. The converse of the  contrapositive of the statement above is also true meaning that if n is prime then $2^n-1$ is prime as well. So to generate 3 numbers of this form we can simply pick the first 3 primes and observe the results
  \begin{align*}
    &n=2\quad: 2^2-1=3\\
    &n=3\quad: 2^3-1=7\\
    &n=5\quad: 2^5-1=31
  \end{align*}
\end{proof}
%%%%%%%%%%%%%%%%%%%%%%%%%%%%%%%%%%%%%%%%%%%%%%%%%%
\end{document}
