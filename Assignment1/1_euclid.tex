\documentclass[11pt]{article} 
\usepackage{amssymb,latexsym,enumerate,graphicx}
\usepackage[sc]{mathpazo}
\usepackage[T1]{fontenc}

\hoffset=0in 
\voffset=0in
\oddsidemargin=0in
\evensidemargin=0in
\topmargin=0in 
\textwidth=6.5in
\textheight=8.6in

\def\Z{\mathbb{Z}}
\def\sage{{\tt sage} }

\pagestyle{empty}

\begin{document}
\setlength{\parindent}{0pt}
\setlength{\parskip}{0.2cm}

$\mbox{}$
\vspace{-1in}

{\sc 2022 Math 310: Number Theory}
\hfill
{\tt https://matthbeck.github.io/310.html}

\vspace{.3in}

\begin{center}
\Large{Worksheet 1: Euclidean Algorithm}
\end{center}

\begin{enumerate}

\item In your group, remind each other about tests for divisibility by 2, 3, and 5.
Prove that these tests work.

\item Let $a, b, c \in \Z$ with $c \ne 0$.
Prove that if $c \mid a$ and $c \mid b$ then $c \mid (ax+by)$ for any $x, y \in \Z$. 

\item 
\begin{enumerate}[(a)]
  \item Why is the fraction $\frac{a}{0}$ ``undefined'' for $a \ne 0$?
  \item Why is $\frac{0}{0}$ ``indeterminate?''
\end{enumerate}

\item The \emph{division algorithm} says that every division problem has a unique quotient and remainder.
Come up with a precise mathematical statement for the division algorithm and prove it.

\item Come up with a definition of the \emph{greatest common divisor} of two integers.
There are various ways to define the gcd; discuss advantages and disadvantages in your group.

\item Pick two 3-digit positive integers $a > b$ and run the division algorithm when $b$ is divided into $a$.
Run the algorithm again when the remainder is divided into $b$; repeat until you get remainder 0.
What are you computing? Why?

\item Let $a, b \in \Z$, not both zero.
Prove that there exist $x, y \in \Z$ such that
\[
  ax+by = \gcd(a,b) \, .
\]
More generally, prove that
\[
  ax+by = c
\]
has a solution $(x,y) \in \Z^2$ if and only if $\gcd(a,b) \mid c$.

\item Andrews 2.3.1.

\item Experiment with the \sage command {\tt divmod}.
Use it with two arguments, say a 6-digit and a 3-digit number, and check that \sage gives the correct answer.

\item Experiment with the \sage command {\tt xgcd}.
Use it with two 5-digit arguments and check that \sage gives the correct answer.

\item Write down a precise statement for each definition we have given this week.
For each definition, give an example and a non-example.

\end{enumerate}

\end{document}

