\documentclass[11pt]{article}
\usepackage{amssymb,latexsym,enumerate,graphicx}
\usepackage[sc]{mathpazo}
\usepackage[T1]{fontenc}

\hoffset=0in
\voffset=0in
\oddsidemargin=0in
\evensidemargin=0in
\topmargin=0in
\textwidth=6.5in
\textheight=8.6in
\def\Z{\mathbb{Z}}
\def\sage{{\tt sage} }

\pagestyle{empty}

\begin{document}
\setlength{\parindent}{0pt}
\setlength{\parskip}{0.2cm}

$\mbox{}$
\vspace{-1in}

{\sc 2022 Math 310: Number Theory}
\hfill
{\tt https://matthbeck.github.io/310.html}

\vspace{.3in}




\begin{center}
\Large{Worksheet 1: Euclidean Algorithm}
\end{center}

\begin{enumerate}

\item In your group, remind each other about tests for divisibility by 2, 3, and 5.
Prove that these tests work.\\

  Proof for the divisibility of three statement:
  Let $n \in \Z$ be expressed in its base 10 representation:
  \[
    n=a_k10^k+a_{k-1}10^{k-1}+\cdots+a_110+a_010^0=\sum_{i=0}^ka_i10^i
  \]
  Argument: if $$3| \sum_{i=0}^ka_i \rightarrow  3|\sum_{i=0}^ka_i10^i$$\\
  \[
    \sum_{i=0}^ka_i10^i=\sum_{i=0}^ka_i10^i
  \]
  \[
    \sum_{i=0}^ka_i10^i=\sum_{i=0}^ka_i(1+10^i-1)
  \]
  \[
    \sum_{i=0}^ka_i10^i=\sum_{i=0}^ka_i+\sum_{i=0}^ka_i(10^i-1)
  \]\\\\
Since $3| \sum_{i=0}^ka_i$ this implies $3| \sum_{i=0}^ka_i=3m,m\in \Z$\\
Also note that: \[
  \forall i\in \Z \quad 3|(10^i-1)
\]
Since \[
  (10^i-1)=\sum_{k=0}^{i-1}9(10^k)=3(\sum_{k=0}^{i-1}3(10^k))=3n,n \in \Z
\]
Using these s we can now express the $(10^i-1)$ as a multiple of 3 in the following way and subsitute the sum of the coeffients by the original argument:

\[
  \sum_{i=0}^ka_i10^i=3m+\sum_{i=0}^ka_i3n
\]
\[
    \sum_{i=0}^ka_i10^i=3(m+\sum_{i=0}^ka_in)
\]
Thus we have shown that $\sum_{i=0}^ka_i10^i$ must also be divisible by three.

\item Let $a, b, c \in \Z$ with $c \ne 0$.
Prove that if $c \mid a$ and $c \mid b$ then $c \mid (ax+by)$ for any $x, y \in \Z$.\\\\
\begin{proof}
  Since c|a this implies that $a=ck,k \in \mathbb{Z}$ likewise since c|b this implies $b=cm, m\in \mathbb{Z}$. Using these new definitions we can re express the term $(ax+by)$ as \[
    (ck(x)+cm(y))=c(kx+my)
  \]
  Since $k,x,m,y$ are integers this means that $c|(ax+by)$ as $(ax+by)$ can be expressed as a multiple of c.
\end{proof}
\\\\
\item
\begin{enumerate}[(a)]
  \item Why is the fraction $\frac{a}{0}$ ``undefined'' for $a \ne 0$?\\\\
  Since the product of any number and zero is zero there is no defined solution for the inverse of multiplication which is division. Assigning a value $\alpha$ to $\frac{a}{0}$ would be akin to making the argument that a*0 = $\alpha$. The mapping of a number to zero under multiplcation is a surjective function, therefore because is is not bijective there does not exist an inverse. In this case that inverse would be division. \\\\
  \item Why is $\frac{0}{0}$ ``indeterminate?''
  \\\\
  $\frac{0}{0}$ is indeterminate because it can represent infinite different answers. For example consider the case in which   $\frac{0}{0}$ assumes a finite value $\beta$. This implies that \[
    0(\beta)=0
  \]
  this statement is true but due to the nature of multiplying by zero the selection of beta is arbitrary which implies that the solution to $\frac{0}{0}$ is also arbitary and can assume any value.
  \\\\
\end{enumerate}

\item The \emph{division algorithm} says that every division problem has a unique quotient and remainder.
Come up with a precise mathematical statement for the division algorithm and prove it.

\begin{proof}
  Assuming that the divison problem this question is concerned with is dealing with integers then the argument of the divison algorithm states that there exists two unique integers q, and r representing the quotient and remainder of the divison. \\
  Some notes:\\
   - The integer divisor cannot be equal to zero as explained in problem 3. \\
   - The remainder must be greater than or equal to zero. \\
   - It does not make sense to have the remainder be greater than or equal to the divisor of the division operation occuring since it would be divided into a smaller remainder so it must be less than\\
   - The resulting quotient and remainder can be used to re-express the dividend.
   \\\\
   Expressed symbolically:
   Given two integers n and m $\in \mathbb{Z}, m \neq 0$
   \[
     \exists ! \quad q,r \in \mathbb{Z} : n= qm+r, 0 \leq r <b
   \]
   To prove this statement we will have to prove both existence and uniqueness.
   \textbf{Existence}:
   Given that n and m are integers the operation of $\frac{n}{m}$
   \\\\
\end{proof}

\item Come up with a definition of the \emph{greatest common divisor} of two integers.
There are various ways to define the gcd; discuss advantages and disadvantages in your group.

\item Pick two 3-digit positive integers $a > b$ and run the division algorithm when $b$ is divided into $a$.
Run the algorithm again when the remainder is divided into $b$; repeat until you get remainder 0.
What are you computing? Why?

\item Let $a, b \in \Z$, not both zero.
Prove that there exist $x, y \in \Z$ such that
\[
  ax+by = \gcd(a,b) \, .
\]
More generally, prove that
\[
  ax+by = c
\]
has a solution $(x,y) \in \Z^2$ if and only if $\gcd(a,b) \mid c$.

\item Andrews 2.3.1.

\item Experiment with the \sage command {\tt divmod}.
Use it with two arguments, say a 6-digit and a 3-digit number, and check that \sage gives the correct answer.

\item Experiment with the \sage command {\tt xgcd}.
Use it with two 5-digit arguments and check that \sage gives the correct answer.

\item Write down a precise statement for each definition we have given this week.
For each definition, give an example and a non-example.

\end{enumerate}

\end{document}
