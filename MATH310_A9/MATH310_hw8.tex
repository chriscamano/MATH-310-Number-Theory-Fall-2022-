\documentclass[11pt]{article}
\usepackage{amssymb,latexsym,amsmath,amsthm,graphicx, cite}
\author{Chris Camano: ccamano@sfsu.edu}
\title{MATH 310  Homework 9? }
\date

\usepackage[many]{tcolorbox}
\tcbset{breakable}
\usepackage{mathptmx}
\usepackage{multirow}
\usepackage{float}
\restylefloat{table}
\hoffset=0in
\voffset=-.3in
\oddsidemargin=0in
\evensidemargin=0in
\topmargin=0in
\textwidth=6.5in
\textheight=8.8in
\marginparwidth 0pt
\marginparsep 10pt
\headsep 10pt

\theoremstyle{definition}  % Heading is bold, text is roman
\newtheorem{theorem}{Theorem}
\newtheorem{corollary}{Corollary}
\newtheorem{defn}{Definition}
\newtheorem{example}{Example}
\newtheorem{proposition}{Proposition}
\usepackage{listings}
\usepackage{xcolor}
\lstset { %
    language=C++,
    backgroundcolor=\color{black!5}, % set backgroundcolor
    basicstyle=\footnotesize,% basic font setting
}


\newcommand{\Z}{\mathbb{Z}}
\newcommand{\N}{\mathbb{N}}
\newcommand{\Q}{\mathbb{Q}}
\newcommand{\R}{\mathbb{R}}
\newcommand{\C}{\mathbb{C}}
\newcommand{\lcm}{\mathrm{lcm}}
\setlength{\parskip}{0cm}
%\renewcommand{\thesection}{\Alph{section}}
\renewcommand{\thesubsection}{\arabic{subsection}}
\renewcommand{\thesubsubsection}{\arabic{subsection}.\arabic{subsubsection}}
\bibliographystyle{amsplain}

%\input{../header}
\newcommand{\bigline}{\\\noindent\makebox[\linewidth]{\rule{\textwidth}{0.4pt}}\\}

\newcommand{\block}[2]{\begin{tcolorbox}[title={#1}]{#2}\end{tcolorbox}}
\begin{document}
\maketitle
\block{Question 1:Andrews 3.2.2}{
Prove that if 12$|n^2-1$ if gcd(n,6)=1
}
\begin{proof}
We first remind ourselves of Fermat's little theorem:
\begin{center}
  If p is a prime and n is an integer then $p|p^2-n $
\end{center}
If 12$|n^2-1$, then by definition of divisibility we know that:
\[
  n^2-1=12k, k\in \Z
\]
So we wish to show that $n^2-1$ is a multiple of 12. \\
We now leverage the gcd statment given to us to combine with this observation: If n and 6 are coprime then this implies that n is 1 less or more than a multiple of 6 six since 6 is not relativley prime to integers 2,3,4 but is for 5 and 7 .

Using this fact we know that in general n is going to assume the form:
\[
  6l+1, \text{ or } 6l-1 \rightarrow 6l \pm 1, l \in \Z
\]
\begin{align*}
  &(6l \pm 1)^2-1\\
  &36l^2 \pm 12l +1 -1 \\
  &12(3l^2\pm l)\\
  &\therefore 12|n^2-1
\end{align*}
If I was really feeling it I would split this in cases instead of preserving the plus minus sign but Ill let you fill in the blanks
\end{proof}
\block{Question #2: Andrews 4.1.2}{
Do there exist integers x such that
\begin{enumerate}
  \item $6x \equiv 5(\mod 4)$\\
  \item $10x \equiv 8(\mod 6)$\\
  \item $12x \equiv 9(\mod 6)$\\
\end{enumerate}
}
\begin{proof}
  $\newline$
\begin{enumerate}
  \item \\
  gcd(6,4)=2 however note that : $4\nmid 5$ so by theorem 5-1 there are no solutions
  \item \\
  gcd(10,6)=2, $2|8$ so there are 4 unique solutions to this congruence by theorem 5-1
  \item \\
  gcd(12,6)=6, $6\nmid 9$ so there are no solutions to this congruence by theorem 5-1.
\end{enumerate}
\end{proof}
\block{Question #3: Andrews 7.1.6}{
Find all primitive roots modulo 5, modulo 9, modulo 11, modulo 13, and modulo 15
}
\begin{proof}
  $\newline$
\begin{enumerate}
  \item 5\\
  We start by computing $\phi(5)=4$ so the question stands:
  Does there exist an a$\in \Z_5$ such that $a^n\equiv 1 \mod 5, n< \phi(5)$? Here 5 is coprime to the elements of $\Z_5$ so we need to consider all elements. \\
  By theorem 7-5 there should be $\phi(\phi(5))=2$ primitive roots. \\\\
  The two primite roots of 5 are :  2 and 3
  \item 9\\
    By theorem 7-5 there should be $\phi(\phi(9))=2$ primitive roots\\\\
    The two primitive roots of 9 are: 2,5
  \item 11\\
    By theorem 7-5 there should be $\phi(\phi(11))=4$ primitive roots\\\\
    The four primitive roots of 11 are :2,6,7,8
  \item 13 \\
    By theorem 7-5 there should be $\phi(\phi(13))=4$ primitive roots\\\\
    The four primitive roots of 13 are : 2,6,7,11
  \item 15\\
    15 does not have primitive roots, evaluating the reduced residue system:
    $\phi(15)=8$\\
    \begin{align*}
      &2^4 \mod 15=1\\
      &4^2 \mod 15=1\\
      &7^4 \mod 15=1\\
      &8^4 \mod 15=1\\
      &11^2 \mod 15=1\\
      &13^4 \mod 15=1\\
      &14^2 \mod 15=1\\
    \end{align*}
\end{enumerate}
\end{proof}
Code I wrote to generate solutions (c++):
\begin{lstlisting}
int gcd(int a, int b)
{
	if (a == 0)
		return b;
	return gcd(b % a, a);
}
int phi(int n) {
	unsigned int result = 1;
	for (int i = 2; i < n; i++)
		if (gcd(i, n) == 1)
			result++;
	return result;
}
void findPrimitiveRoots(int n) {
	cout << "________Primitive Root Finder________" << endl;
	vector<int> rrs;
	for (int i = 2; i < n; i++) {
		if (gcd(i, n) == 1) {
			rrs.push_back(i);
		}
	}

	for (int i = 0; i < rrs.size(); i++) {
		bool is_not_root = false;
		for (int j = 2; j < phi(n); j++) {
			if (int(pow(rrs[i], j)) % n == 1) {
				is_not_root = true;
				double b = pow(rrs[i], j);
				cout << endl;

				cout << rrs[i] << " is not a primitive root " << endl;

				cout << "Remainder " << (int(b)) % n << " for " <<
                      rrs[i] << "^" << j << endl;
				cout << pow(rrs[i], j) << " mod " << n << "=" <<
                      (int(b)) % n << endl;

				break;
			}
		}
		if (!is_not_root) {
			cout << endl;

			cout << rrs[i] << " is a primitive root" << endl;

		}
	}
}
\end{lstlisting}
\block{Question #4: Andrews 7.2.15}{
How many primitive roots exist for the moduli 6,7,8,9,10?
}
\begin{proof}
\begin{enumerate}
  \item \\
  \begin{align*}
    &\phi(\phi(6))\\
    &\phi(2)=1
  \end{align*}
  So by theorem 7-5 there is 1 primitive root
  \item \\
  \begin{align*}
    &\phi(\phi(7))\\
    &\phi(6)=2
  \end{align*}
  So by theorem 7-5 there are 2 primitive roots
  \item \\
  8 has no primitive roots proof below
  \begin{align*}
    3^2\equiv 1\mod8\\
    5^2\equiv 1\mod8\\
    7^2\equiv 1\mod8
  \end{align*}
  \item \\
  \begin{align*}
    &\phi(\phi(9))\\
    &\phi(6)=2
  \end{align*}
  So by theorem 7-5 there are 2 primitive roots
  \item \\
  \begin{align*}
    &\phi(\phi(10))\\
    &\phi(4)=2
  \end{align*}
  So by theorem 7-5 there  2 primitive roots
\end{enumerate}

\end{proof}
\block{Question #5: Stein 2.23}{
Find all four solutions to the equation : \
\[
    x^2-1\equiv 0 \mod 35
\]
\[
    x^2\equiv 1 \mod 35
\]
}
Solutions: 1,6,29,34


\begin{proof}
  \textbf{proof by being a computer scientist}\\

  Code I wrote to generate solutions (c++):
  \begin{lstlisting}
  for (int i = 1; i < 35; i++) {
  			if (int(pow(i, 2)) % 35 == 1) {
  				cout << i << " is a solution " << endl;
  			}
  	}
    \end{lstlisting}
\end{proof}
A more "theoretic" proof:
\begin{proof}
  We can start by splitting congruence as follows:
  \[
      x^2\equiv 1 \mod 5
  \]
  \[
      x^2\equiv 1 \mod 7
  \]
  if remove the sqaure root we end up generating four sub problems which luckily alligns with out problem description:
  \begin{align*}
    &x\equiv 1 \mod 7 \\
    &x\equiv 1 \mod 5\\
  \end{align*}
  Here we see very clearly that the solution to this system is just 1
  \begin{align*}
    &x\equiv -1 \mod 7 \rightarrow x\equiv 6 \mod7\\
    &x\equiv -1 \mod 5\rightarrow x\equiv 4 \mod5\\
  \end{align*}
  Here we see the solution of 34
  \begin{align*}
    &x\equiv -1 \mod 7 \rightarrow x\equiv 6 \mod7 \\
    &x\equiv 1 \mod 5\\
  \end{align*}
  Next we can see by inspection the solution of 6
  \begin{align*}
    &x\equiv 1 \mod 7 \\
    &x\equiv -1 \mod 5\rightarrow x\equiv 4 \mod5\\
  \end{align*}
  finally our last solution is 29
\end{proof}
\end{document}
